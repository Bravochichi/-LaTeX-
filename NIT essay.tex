%----------------南昌工程学院毕业论文LaTeX模板------------------------
%----------------AUTHOR  :  GUANGCHUN-------------------------

\documentclass[UTF8,twoside]{ctexart} % use larger type; default would be 10pt
\usepackage{graphicx}
\usepackage {setspace}%设置正文行间距为1.5倍
\usepackage[a4paper,top=25mm, bottom=20mm, left=25mm, right=20mm]{geometry}%设置页边距
\usepackage{graphicx}%插入图片
\usepackage{booktabs} %美观的表格
\usepackage{amsmath}%数学模式
\usepackage{fancyhdr} %页面布局
\usepackage{titletoc}  %与标题目录有关
\usepackage[square]{natbib}%文献系统
\usepackage[bookmarks=true,colorlinks,linkcolor=black]{hyperref} %这是可选可不选的宏包,方便PDF软件的阅读
%\usepackage{appendix}%附录部分,有的话取消注释

%--------------------------------
%目录样式
%目录页设置
\renewcommand{\contentsname}{\zihao{2}\songti 目\quad 录}
\titlecontents{section}[0em]{\heiti\zihao{4} }{\thecontentslabel\ }{}
{\hspace{.5em}\titlerule*[4pt]{$\cdot$}\contentspage}
\titlecontents{subsection}[2em]{\vspace{0.1\baselineskip}\zihao{-4}}{\thecontentslabel\ }{}
{\hspace{.5em}\titlerule*[4pt]{$\cdot$}\contentspage}
\titlecontents{subsubsection}[4em]{\vspace{0.1\baselineskip}\zihao{-4}}{\thecontentslabel\ }{}
{\hspace{.5em}\titlerule*[4pt]{$\cdot$}\contentspage}
%--------------------------

%--------------------------
%三级标题格式
%\usepackage{sub4section}
%\secnumdepth{4}
\ctexset {
	section = {
name = {第,章},
number = \chinese{section},
nameformat = \songti \bfseries\zihao{3},
titleformat = \centering\songti \bfseries\zihao{3},
}
}
\ctexset{
	subsection = {
%format = \raggedright ,
%nameformat = \bfseries\zihao{4},
format = \heiti\bfseries\zihao{4},
}
}
\ctexset{
	subsubsection = {
format = \raggedright ,
nameformat = \bfseries\zihao{-4},
titleformat =\songti \bfseries\zihao{-4},
}
}
\setcounter{secnumdepth}{5}
\ctexset{
	paragraph = {	
	name = {,.},
        number=\arabic{paragraph}, 
        format=\indent{}\songti\zihao{-4}
}
}
\ctexset{
	subparagraph = {
	name = {(,)},
	number = \arabic{subparagraph},
	 format=\songti\zihao{-4}

}
}

%---------------标注样式-----------	
\newcommand{\upcitep}[1]{\textsuperscript{\textsuperscript{\citep{#1}}}}  
\setcitestyle{numbers}
%---------------------------
%代码背景样式
\RequirePackage{listings}
\RequirePackage{xcolor}
\definecolor{dkgreen}{rgb}{0,0.6,0}
\definecolor{gray}{rgb}{0.5,0.5,0.5}
\definecolor{mauve}{rgb}{0.58,0,0.82}
\lstset{
	numbers=left,  
	frame=tb,
	aboveskip=3mm,
	belowskip=3mm,
	showstringspaces=false,
	columns=flexible,
	framerule=1pt,
	rulecolor=\color{gray!35},
	backgroundcolor=\color{gray!5},
	basicstyle={\ttfamily},
	numberstyle=\tiny\color{gray},
	keywordstyle=\color{blue},
	commentstyle=\color{dkgreen},
	stringstyle=\color{mauve},
	breaklines=true,
	breakatwhitespace=true,
	tabsize=3,
}

%-------------------其他设置-----------------------------
\renewcommand{\thefigure}{\arabic{section}-\arabic{figure}}  %设置图片标题样式
\renewcommand{\thetable}{\arabic{section}-\arabic{table}}  %设置表格标题样式
\renewcommand{\theequation}{\arabic{section}-\arabic{equation}} %设置公式右边注释样式

%-----页眉页脚设置--------
\pagestyle{fancy}
\fancyhf{}
\fancyhead[CO]{\songti \zihao{-5}南昌工程学院本(专)科毕业设计(论文) }  %奇数页页眉设置
\fancyhead[CE]{\songti \zihao{-5}\rightmark}   %偶数页页眉设置
\renewcommand\sectionmark [1]{\markboth  {}{第 \thesection 章 : #1}}
\cfoot{\thepage}

%--------论文主体--------
\begin{document}
%--------封面设计-------------
\begin{titlepage}	
		\vspace*{0.5cm}
		\centering		
		{\heiti\zihao{-1} 南昌工程学院本(专)毕业设计}		
		\centering
	{ \songti\zihao{4}
		\vspace*{1.5cm}
		\quad\includegraphics[width=5.5cm,height=5.5cm]{figure/nit.jpg}\\{}		
		 \vskip 2cm
		 \makebox[50mm]{~~系~ (院)}
		 \underline{\makebox[75mm][c]{ 这里填写}}\\%院系
		 \vskip 0.9cm
		 \makebox[50mm]{专\qquad 业}
		 \underline{\makebox[75mm][c]{ 这里填写}}\\%专业
		 \vskip 0.9cm
		 \makebox[50mm]{毕业设计(论文)题目}
		 \underline{\makebox[75mm][c]{ 这里填写}}\\%题目
		 \vskip 0.9cm
		 \makebox[50mm]{学生姓名}
		 \underline{\makebox[75mm][c]{ 这里填写}}\\%姓名
		 \vskip 0.9cm
		 \makebox[50mm]{班\qquad  级}
		 \underline{\makebox[75mm][c]{ 这里填写}}\\%班级
		 \vskip 0.9cm
		 \makebox[50mm]{学\qquad 号}
		 \underline{\makebox[75mm][c]{ \LARGE 2016214000}}\\%学号
		 \vskip 0.9cm
		 \makebox[50mm]{指导老师}
		 \underline{\makebox[75mm][c]{ 这里填写}}\\%老师
		 \vskip 0.9cm
		  \makebox[50mm]{完成日期}
		 \underline{\makebox[75mm][c]{ 这里填写}}\\%日期
		 \vskip 1cm
		 \LARGE \heiti{\number \year }~年~\heiti{\number\month}~月~\heiti{\number\day}~日	
	}	 
\end{titlepage}
%---------论文题目---------------
\newpage\thispagestyle{empty}	
		\rightline{\quad\includegraphics[width=5cm,height=1cm]{figure/picture1.jpg}}
	\vskip 2cm
	\centering

         {\songti  \bfseries \zihao{2} 基于实体建模的数控仿真系统环境的开发  } %标题
         \vskip 0.5cm
        \rightline {\songti \zihao{-2} 小标题}%如果需要小标题的话,没有需要请注释
        \vskip 2.5cm

        { \bfseries \Large The development of Environment for NC Simulation system  based on the solid modelling}%英文标题
        \vskip 13cm
   \begin{flushleft}
        \songti\zihao{4}
        总计\makebox[50mm]{毕业设计(论文)}  \underline{\makebox[25mm][c]{ X}} 页\\
       \quad \quad \makebox[50mm]{表\qquad \quad \qquad 格}  \underline{\makebox[25mm][c]{ X}} 个\\
        \quad \quad \makebox[50mm]{插\qquad \quad\qquad 图}  \underline{\makebox[25mm][c]{ X}}幅\\
   \end{flushleft}
\pagenumbering{Roman} %页码采用罗马数字
\include{tex/chinese-abstract}
\include{tex/english-abstract}

%\setcounter {tocdepth}{3}
%--------开始用阿拉伯数字编码-----
\pagenumbering{arabic}
%--------目录---------------
\tableofcontents\thispagestyle{plain}
%-------第一章-------------
\newpage
\raggedright
\setlength {\parindent }{2 em}

\onehalfspacing %正文1.5倍间距
\songti \zihao{-4} %正文采用宋体小四号
\section{一级标题三号宋体加粗居中}   
\subsection{二级标题四号黑体加粗居左}
\subsubsection{三级标题小四号宋体加粗居左}
\par 正文部分用小四号字。……

\include{tex/section2}
%-------第三章---------------------
\newpage
\section{程序代码}
\begin{lstlisting}[language=C++,escapeinside=``]
#include<iostream>
using namespace std;
int main
{
	cout<<"Hello world!"<<endl;//`输出`
	return 0;
}
\end{lstlisting}

%-------参考文献---------------------
\newpage
\bibliographystyle{plainnat}
\bibliography{bib/mybib}
\addcontentsline{toc}{section}{参考文献} %添加到目录

%\backmatter %在此命令后恢复成罗马数字页码
%-----------------------------------------
\end{document}
